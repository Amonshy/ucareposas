\section{Carga inicial}


\begin{frame}{Carga inicial}
 \begin{block}{Formato del archivo de datos iniciales}
 \begin{itemize}
  \item Los datos se tienen que redactar siguiendo un determinado orden.
  \item La información se tiene que escribir entre comillas simples y separada por ;.
  \item Se escribe un punto entre deporte y deporte para así poder diferenciarlos.
  \item Si se necesita tener más de un dato para un mismo atributo se escriben separados por comas.
  \item Si se quiere prescindir de un atributo se deja vacío el espacio reservado para dicho atributo.
  \item El archivo debe terminar con un salto de línea.
 \end{itemize}
 \end{block}
\end{frame}

\begin{frame}{Carga inicial}
 \begin{block}{Lectura de los datos y creación del archivo que contiene la información con la sintaxis correcta (archivo entrada\_datos.pl)}
 \begin{description}
  \item [Predicado principal:]se encarga de abrir el fichero, crear el nuevo archivo, llamar al predicado procesar\_fichero y cerrar ambos ficheros.
  \item [Predicado procesar\_fichero:]se encarga de leer cada línea del archivo y escribirla correctamente en el nuevo fichero.
  \item [Predicados auxiliares(leer\_nombre,leer\_categoria\ldots):]son una serie de predicados que utiliza procesar\_fichero para leer y escribir atributo a atributo.
 \end{description}
 \end{block}
\end{frame}

\begin{frame}{Carga inicial}
 \begin{block}{Ejemplo de entrada}
'Bodyboard';'acuático';;'tabla';'1';;'Bodyboard';.
 \end{block}

 \begin{block}{Ejemplo de salida}
deporte('Bodyboard'):- categoría('acuático'),accesorio('tabla'),
		jugadores('1'),nombre\_deporte('Bodyboard').
 \end{block}
\end{frame}


\begin{frame}{Carga inicial}
 \begin{block}{Creación de la base de conocimiento}
La base de conocimiento (deportes.pl) incluye:
\begin{itemize}
 \item El predicado objetivo.
 \item Los datos que se han obtenido al ejecutar entrada\_datos.pl.
 \item Los predicados que preguntan por cada atributo de manera concreta.
\end{itemize}
 \end{block}
\end{frame}
