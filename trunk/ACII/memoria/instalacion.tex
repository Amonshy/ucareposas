\section{Instalación}
\subsection{GNU/Linux}
\subsubsection{Drivers}
Lo primero que tenemos que hacer es descargar los ``Developer Drivers'' de la siguiente dirección
  \begin{verbatim}
   http://developer.nvidia.com/object/cuda_3_0_downloads.html
  \end{verbatim}
Nos descargamos el los drivers que correspondan con nuestra versión de GNU/Linux.
Para instalarlo tenemos que cambiar a una consola de sistema pulsando ctrl+alt+f2, por ejemplo. Una vez hecho esto
nos logueamos y terminamos el entorno gráfico de la siguiente forma (como superusuario):
\begin{description}
 \item[KDE] \comando{/etc/init.d/kdm stop}
 \item[GNOME] \comando{etc/init.d/gdm stop}
 \end{description}
Ahora podemos proceder con la instalación del driver ejecutando como superusuario lo siguiente
  \begin{lstlisting}[style=consola]
  sh NombreDelDriver.run
  \end{lstlisting}
Lo único que nos queda por hacer es seguir los pasos del asistente.

\subsubsection{CUDA Toolkit}
Para instalar las herramientas de compilación ejecutamos lo siguiente como superusuario
  \begin{lstlisting}[style=consola]
  sh NombreDelToolkit.run
  \end{lstlisting}

\subsubsection{SDK}
También podemos instalar el SDK que contiene ejemplos y documentación sobre CUDA, pero esto es opcional.
\subsubsection{Configuración}
Para poder utilizar el compilador \comando{nvcc} es necesario indicar donde se encuentra, para ello creamos el
archivo\\
\comando{/etc/profile.d/cuda.sh}\\
con el siguiente contenido
  \begin{lstlisting}[style=BASH]
  #!/bin/sh
  export PATH=$PATH:/usr/local/cuda/bin
  \end{lstlisting}

Solo nos queda indicar dónde se encuentran las bibliotecas para poder ejecutar los programas una vez compilados,
para ello creamos el archivo\\
\comando{/etc/ld.so.conf.d/libcuda.conf}\\
con el siguiente contenido\\
\comando{/usr/local/cuda/lib}

\subsection{Windows}
\subsubsection{Drivers}
El primer paso que debemos seguir es descargar los últimos ``Developer Drivers''. Los podemos encontrar en la página
  \begin{verbatim}
   http://developer.nvidia.com/object/cuda_3_0_downloads.html
  \end{verbatim}
Seleccionaremos Windows y el driver el adecuado según nuestra
versión, diferenciando si es de 32 o 64 bits. Una vez descargados procedemos con la instalacíon siguiendo
los pasos que nos proporciona el asistente.
\subsubsection{CUDA Toolkit}
Lo siguiente que tenemos que hacer es descargar e instalar ``CUDA Toolkit'' que contiene las herramientas necesarias
(como el compilador nvcc) para compilar un programa de CUDA. Para ello procedemos de forma analoga a la sección anterior.
\subsubsection{SDK}
También podemos instalar el SDK que contiene ejemplos y documentación sobre CUDA, pero no es necesario.
\subsubsection{Microsoft Visual Studio}
CUDA en Windows utiliza el compilador \comando{cl.exe} de Microsoft Visual Studio, por lo que deberemos instalarlo.
Si no disponemos de ninguna licencia podemos obtener la versión Express que es gratuita y suficiente para nuestro propósito,
aunque tendremos que registrarla, de forma gratuita también. Actualmente existen dos versiones Express, la 2005 y la 2010,
pero la 2010 no está todavía soportada por CUDA, así que instalaremos la 2005. Para ello tenemos que entrar en la
siguiente página web:
  \begin{verbatim}
   http://msdn.microsoft.com/es-es/express/aa975050.aspx
  \end{verbatim}
Buscamos Visual C++ 2005 Express Edition y seleccionamos nuestro idioma. Para instalarlo basta con seguir los pasos
del asistente
\subsubsection{Configuración}
Lo que tenemos que hacer ahora es incluir el compilador \comando{cl.exe} en el path del sistema, para ello
seguiremos los siguientes pasos.

Primero haremos click con el botón derecho del ratón sobre Mi PC y luego en propiedades.
\figura{pol1.png}{scale=0.5}{Paso 1}{paso1}{H}
En segundo lugar seleccionaremos la pestaña ``Opciones avanzadas'' y haremos click en ``Variables de entorno''.
\figura{pol2.png}{scale=0.5}{Paso 2}{paso2}{H}
En la sección ``Variables del sistema'' buscaremos la variable ``Path'' y haremos click en ``Modificar''.
\figura{pol3.png}{scale=0.5}{Paso 3}{paso3}{H}
Añadiremos al final del campo ``Valor de variable'', separado con punto y coma, la ruta completa del compilador
del Visual Studio (\comando{cl.exe}).
\figura{pol4.png}{scale=0.5}{Paso 4}{paso4}{H}
Una vez hecho todo esto el sistema estara preparado para compilar nuestro primer programa.