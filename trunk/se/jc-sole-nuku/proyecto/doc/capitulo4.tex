\chapter{Contrucción del shell}

El shell utilizado es el mismo que se usó en la práctica de las aves, que contiene la siguiente estructura:

\begin{itemize}
\item Separación de esqueleto y conocimiento
	\begin{itemize}
	\item Inclusión de predicados genéricos, según los objetivos, en la base de conocimientos (\emph{objetivo(Deporte) :- deporte(Deporte).}).
	\item Incluir en shell un predicado que resuelva dicho objetivo.
	\end{itemize}

\item Construcción del intérprete de órdenes
	\begin{itemize}
	\item Dar la bienvenida.
	\item Cargar una base del conocimiento.
	\item Permitir realizar consultas.
	\item Mostrar un menú de ayuda.
	\item Listar los hechos ya respondidos por el usuario.
	\item Finalizar la ejecución.
	\end{itemize}

\item Desarrollo del subsistema de explicaciones
	\begin{itemize}
	\item Los únicos predicados a invocar son menú y preguntar.
	\item Historizar debe incluir un histórico, para poder consultarlo.
	\item Se mejora el prototipo para conseguir las explicaciones.
	\end{itemize}
\end{itemize}


