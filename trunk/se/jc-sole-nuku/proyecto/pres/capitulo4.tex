\section{Construcción del shell}
\begin{frame}
 \tableofcontents[currentsection,hideothersubsections]    
\end{frame}

\begin{frame}{Construcción del shell}
 \begin{block}{Construcción del shell}
  \begin{itemize}
   \item Mismo shell que el visto en clase.
   \item Modificación del predicado que muestra la ayuda.
  \end{itemize}
 \end{block}
\end{frame}


\begin{frame}{Construcción del shell II}
 \begin{block}{Separación de esqueleto y conocimiento}
	\begin{itemize}
	\item Inclusión de predicados genéricos, según los objetivos, en la base de conocimientos (\emph{objetivo(Deporte) :- deporte(Deporte).}).
	\item Incluir en shell un predicado que resuelva dicho objetivo.
	\end{itemize}
 \end{block}
\end{frame}


\begin{frame}{Construcción del shell III}
 \begin{block}{Construcción del intérprete de órdenes}
	\begin{itemize}
	\item Dar la bienvenida.
	\item Cargar una base del conocimiento.
	\item Permitir realizar consultas.
	\item Mostrar un menú de ayuda.
	\item Listar los hechos ya respondidos por el usuario.
	\item Finalizar la ejecución.
	\end{itemize}
 \end{block}
\end{frame}


\begin{frame}{Construcción del shell IV}
 \begin{block}{Desarrollo del subsistema de explicaciones}
	\begin{itemize}
	\item Los únicos predicados a invocar son menú y preguntar.
	\item Historizar debe incluir un histórico, para poder consultarlo.
	\item Se mejora el prototipo para conseguir las explicaciones.
	\end{itemize}
 \end{block}
\end{frame}
